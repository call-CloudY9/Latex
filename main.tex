\documentclass[a4paper,12pt]{article}
\usepackage{cmap}		
\usepackage[utf8]{inputenc}
\usepackage{framed}
\usepackage[english,russian]{babel}
\usepackage{framed}
\usepackage{hyperref}
\usepackage{amsmath}
\usepackage{graphicx}
\usepackage[colorinlistoftodos]{todonotes}
\usepackage{wrapfig}
\usepackage{lipsum}
\usepackage{listings}
\usepackage{minted}
\usepackage{color}
\usepackage{xcolor}
\usepackage{indentfirst}
\usepackage{times}
\usepackage{textcomp}
\usepackage{wrapfig}
\usepackage{adjustbox}
\usepackage{txfonts}
\usepackage{lipsum}
\usepackage{physics}
\usepackage{rotating}
\graphicspath{{../}}


\definecolor{my gray}{rgb}{0.4,0.4,0.4}
\definecolor{my green}{rgb}{0,0.8,0.6}
\definecolor{my orange}{rgb}{1.0,0.4,0}
\definecolor{code green}{rgb}{0,0.6,0}
\definecolor{code gray}{rgb}{0.5,0.5,0.5}
\definecolor{code purple}{rgb}{0.58,0,0.82}
\definecolor{back colour}{rgb}{0.95,0.95,0.92}

\lstdefinestyle{custom}{
  % backgroundcolor=\color{backcolour},
  belowcaptionskip=1\baselineskip,
  break lines=true,
  frame=L,
  xleftmargin=\parindent,
  language=sh,
  showstringspaces=false,
  basic style=\ttfamily\bfseries,
  columns=fixed,
  more keywords={ifconfig,IP,route,dhclient,nmcli,ping},
  keyword style=\bfseries\color{green!40!black},
  comment style=\itshape\color{purple!40!black},
  identifier style=\color{blue},
  string style=\color{purple},
  numbers=left,
  numbersep=12pt,
  number style=\ttfamily\scriptsize\color{my gray},
  tab size=2,
}
\lstset{
    language=Python,          
    breaklines=true,         
    commentstyle=\color{green},   
    keywordstyle=\color{blue},    
    stringstyle=\color{red},      
    basicstyle=\ttfamily\small   
}




\newcommand{\HRule}{\rule{\linewidth}{0.5mm}}

\begin{document}

\begin{titlepage}
\begin{center}

\textsc{\Large Федеральное государственное автономное образовательное учреждение высшего образования}\\
\textsc{\large "Национальный исследовательский университет ИТМО"}\\[0.5cm]

% Upper part of the page. The '~' is needed because \\
% only works if a paragraph has started.
\includegraphics[width=0.5\textwidth]{img/logo.png}~\\[0.5cm]

\textsc{\Large Названние дисциплины: ИСРПО 
\\ Отчет по лабораторной работе №3
\\ Тема работы: "Работа с LaTeX"}
\\[5cm]


% Author and supervisor
\noindent
\begin{minipage}{0.4\textwidth}
\begin{flushleft} \large
\emph{Выполнил:}\\
Рахимкулов \textsc{Д.~Р.}, M3115
\end{flushleft}
\end{minipage}
\begin{minipage}{0.4\textwidth}
\begin{flushright} \large
\emph{Преподаватель:}\\
\quad\quad\quadКарим \textsc{Х.~А.}
\end{flushright}
\end{minipage}

\vfill

% Bottom of the page
{\large 15.10.2024}

\end{center}
\end{titlepage}


\tableofcontents
\newpage

\section{Общее описание библиотеки geometric\_lib}
Библиотека \texttt{geometric\_lib} решает задачи, связанные с геометрическими вычислениями, такими как:
\\- рассчёт параметров геометрических фигур;
\\- обработка и анализ данных, представленных в виде геометрических объектов.

Данная библиотека предоставляет набор функций и классов, позволяющих выполнять вычисления с геометрическими фигурами, включая обработку координат, вычисления площадей и периметров, а также визуализацию результатов.
\\
\section{Описание файлов программ из репозитория}
\subsection{Файл \texttt{square.py}}
\begin{lstlisting}[language=Python]
def area(a):
    return a * a

def perimeter(a):
    return 4 * a
\end{lstlisting}

Данная прграмма вычисляет площадь и пермиетр квадрта по следующей формуле:
\[
S = a^2
\]
\[
P = 4 a
\]

\subsection{Файл \texttt{triangle.py}}
\begin{lstlisting}[language=Python]
def area(a, b, c):
    return ((a + b + c) / 2)
    
def perimeter(a, b, c):
    return (a + b + c)
\end{lstlisting}

Данная программа вычисляет площадь и периметр треугольника по следующей формуле:

\[
S = \frac{1}{2} \cdot (a + b + c)
\]
\[
P = (a + b + c)
\]
\subsection{Файл \texttt{calculate.py}}
\begin{lstlisting}[language=Python]
import circle
import square
figs = ['circle', 'square']
funcs = ['perimeter', 'area']
sizes = {}
def calc(fig, func, size):
	assert fig in figs
	assert func in funcs
	result = eval(f'{fig}.{func}(*{size})')
	print(f'{func} of {fig} is {result}')
if __name__ == "__main__":
	func = ''
	fig = ''
	size = list()
	while fig not in figs:
		fig = input(f"Enter figure name, avaliable are {figs}:\n")
	while func not in funcs:
		func = input(f"Enter function name, avaliable are {funcs}:\n")
	while len(size) != sizes.get(f"{func}-{fig}", 1):
		size = list(map(int, input("Input figure sizes separated by space, 1 for circle and square\n").split(' ')))
	calc(fig, func, size)
\end{lstlisting}
Данный код предоставляет простую оболочку для динамического вычисления значений, основываясь на введённых пользователем данных, и использует функцию eval для вызова методов из внешних модулей, что делает систему гибкой и расширяемой. Это может быть полезно для добавления новых фигур и методов в будущем, при условии соответствующей модификации модулей circle и square. Практическое использование eval следует осуществлять с осторожностью, так как это может привести к потенциальным уязвимостям, если введённые данные не контролируются должным образом.
\newpage
\subsection{Файл \texttt{circle.py}}
\begin{lstlisting}[language=Python]
import math

def area(r):
    return (math.pi * r * r)

def perimeter(r):
    return (2 * math.pi * r)
\end{lstlisting}

Данная программа вычисляет площадь и периметр круга по следующей формуле:

\[
S = \pi \cdot r^2
\]
\[
P = 2 \cdot \pi \cdot r
\]
\\
\\
\\
\section{Ссылки на проект}
\subsection{Исходный код проекта можно просмотреть на GitHub \href{https://github.com/call-CloudY9/geometric_lib.git}{здесь}}
\\
\subsection{Исходный код overleaf можно найти 
\href{https://github.com/call-CloudY9/Latex.git}{здесь}}

\end{document}
